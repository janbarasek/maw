%%% ------------Cut here------------

%% 1.
%% We redefine internal commands of pdfanim. If you use
%% \PDFAnimLoad[remember,loop,startframe=30]{aa}{trac}{90}, it is assumed
%% that the file trac.pdf has at least 90 pages and first 90 pages form
%% the frames of the animation.

%% 2. In running animation, click means pause, not stop! In paused 
%% animation, shift+click plays backwards. The shift button changes the
%% direction also for stepwise animations.

%% Macros start here.

\newcount \mujcounter
\makeatletter

\def\PDFAnim@PictureButtonWidget#1{%
  /Subtype /Widget
  \ifPDFAnim@hidden /F 6 \else /F 4 \fi
  \ifPDFAnim@step
  \ifPDFAnim@checkshift
  /TU(Click for the next frame. Shift+Click for previous frame.)
  \else
  /TU(Click for the next frame.)
  \fi
  \else
  \ifPDFAnim@checkshift
  /TU(Click to play. Shift+Click to play backwards.)
  \else
  /TU(Click to play.)
  \fi
  \fi
  \ifx\PDFAnim@name\@empty
  \else
    /T (\PDFAnim@name)
  \fi
  /FT /Btn
  /Ff 65536
  /H /N
  /BS << /W 1 /S /S >>
  /MK <<
    /TP 1
    /I #1 0 R
    /IF << /SW /\PDFAnim@scale /S /\PDFAnim@scaletype /A [0.5 0.5] >>
    \ifx\PDFAnim@bcolor\@empty
    \else
      /BC [\PDFAnim@bcolor]
    \fi 
    \ifx\PDFAnim@bgcolor\@empty
    \else
      /BG [\PDFAnim@bgcolor]
    \fi 
  >>
  \ifx\PDFAnim@onclick\@empty
  \else
    /A << /S /JavaScript /JS (\PDFAnim@onclick;) >>
  \fi 
}



\def\PDFAnim@PictureButton[#1]#2#3{% parameters, picture file name,page
%  \message{PDFanimpicturebutton #1; #2; #3^^J}
  \bgroup
  \mujcounter=#3
  \advance \mujcounter by 1
    \setkeys{PDFAnim}{#1}%
%     \ifPDFAnim@checkshift
%     \expandafter\xdef\csname checkshift#2\endcsname##1{##1}
%     \else
%     \expandafter\xdef\csname checkshift#2\endcsname##1{}
%     \fi
    \immediate\pdfximage
      \ifx\PDFAnim@width\@empty\else width \PDFAnim@width\fi 
      \ifx\PDFAnim@height\@empty\else height \PDFAnim@height\fi
      \ifx\PDFAnim@depth\@empty\else depth \PDFAnim@depth\fi
      page \mujcounter
      {#2}%
    \def\PDFAnim@Obj{\the\pdflastximage}%
    \PDFAnim@LayoutPictureButtonField{%
      \leavevmode
      \pdfstartlink user{\PDFAnim@PictureButtonWidget{\PDFAnim@Obj}}%
      \ifPDFAnim@hidden
        \phantom{\pdfrefximage\PDFAnim@Obj}%
      \else
        \ifPDFAnim@fallback
	  \pdfrefximage\PDFAnim@Obj%
	\else
          \phantom{\pdfrefximage\PDFAnim@Obj}%
	\fi
      \fi
      \pdfendlink
    }%
  \egroup
}

\renewcommand*{\PDFAnimLoad}[4][]{%
%  \vbox to\z@{\hb@xt@\z@{%
%\message{soubor je #3 ^^J}
\expandafter\edef\csname LastFrame#2\endcsname{#4}
    \@PDFAnimLoad[#1]{#2}{#3}{#4}%\hss}\vss}%
  }

\newif\ifPDFAnim@checkshift
\PDFAnim@checkshifttrue
\define@key{PDFAnim}{noshift}[true]{%
  \PDFAnim@checkshiftfalse
}

  
\renewcommand*{\@PDFAnimLoad}[4][]{% options, name, files, number of files   
  \bgroup
  \def\PDFAnim@defaultframe{0}
  \setkeys{PDFAnim}{#1}%  
  \expandafter\xdef\csname #2PDFAnim@DefaultFrame\endcsname{\PDFAnim@defaultframe}
  \ifPDFAnim@auto
  \expandafter\gdef\csname#2PDFauto\endcsname{Pause}
  \else
  \expandafter\gdef\csname#2PDFauto\endcsname{Play}
  \fi
  \makeatletter
  \expandafter\xdef\csname #2PDFDelayInterval\endcsname{\PDFAnim@interval}
  \ifx\PDFAnim@usecnt\@empty
  \else
  \def\PDFAnim@startframe{\PDFAnim@usecnt cnt}%
  \fi
  \ifx\PDFAnim@use\@empty
  \def\PDFAnim@UseName{#2h}%
  \PDFAnim@argdef{#2@h}{%
    \noexpand\PDFAnimPictureButton[	
    width=0pt,
    height=0pt,
    depth=0pt,,
    name=#2h##1,
    hidden=true
    ]{#3.\PDFAnim@extension}{##1}%
  }%
  \else
  \def\PDFAnim@UseName{\PDFAnim@use h}%
  \PDFAnim@argdef{#2@h}{}%
  \fi 
%   \expandafter\xdef\csname #2@PDFAnimName\endcsname{\PDFAnim@UseName}
%   \ifPDFAnim@reverse
%   \expandafter\xdef\csname#2@PDFAnimReverse\endcsname##1##2{##1}
%   \else
%   \expandafter\xdef\csname#2@PDFAnimReverse\endcsname##1##2{##2}
%   \fi
  \PDFAnim@def{#2}{%
    \noexpand\PDFAnimPictureButton[%
    #1,
    name=#2,
    onclick={
      \ifPDFAnim@noclick\else
      \ifPDFAnim@step
      if (#2running==1) #2PDFStopAnimation();        
      %%% stepwise animation
      % if (#2running == 0) 
      % {
      %   #2running = 1;
      %   #2cnt = \PDFAnim@startframe;
      % }
      \ifPDFAnim@checkshift
      #2eventshift = event.shift;
      \else
      #2eventshift = false;
      \fi
      \ifPDFAnim@reverse  % reverse
      if (#2eventshift) #2lr=true; else #2lr=false;
      \else % normal 
      if (#2eventshift) #2lr=false; else #2lr=true;
      \fi
      #2running=1;
      #2PDFAnimate();
      #2running=0;
      %%% not stepwise animation
      \else % nostep
      #2StartAnimation();
      \fi
      \fi
    }
  ]{#3.\PDFAnim@extension}{\PDFAnim@defaultframe}%
}%
\PDFAnim@def{#2@frames}{#4}
\PDFAnimJS@def{\thePDFAnimNr}{%
  var #2running = 0;
  var #2cnt = \PDFAnim@defaultframe;
  var #2interval = \PDFAnim@interval;
  var #2eventshift = false;
  function #2StartAnimation(){
    var PDFrun=true;
    \ifPDFAnim@checkshift
    #2eventshift = event.shift;
    \else
    #2eventshift = false;
    \fi
    \ifPDFAnim@reverse
    if ((#2running == 0)&&(#2cnt<=0)&&!(#2eventshift))
    {
      #2cnt=(#4-1);
      try {
      this.getField('#2').buttonSetIcon(this.getField('\PDFAnim@UseName'+#2cnt).buttonGetIcon());
      } catch (e) {}
      #2PDFAnimUpdateTextProc(#2cnt);
      PDFrun=false;
    }      
    if ((#2running == 0)&&(#2cnt>=(#4-1))&&(#2eventshift))
    {
      #2cnt=0;
      try {
      this.getField('#2').buttonSetIcon(this.getField('\PDFAnim@UseName'+#2cnt).buttonGetIcon());
      } catch(e) {}
      #2PDFAnimUpdateTextProc(#2cnt);
      PDFrun=false;
    }
    \else
    if ((#2running == 0)&&(#2cnt==0)&&(#2eventshift))
    {
      #2cnt=(#4-1);
      try {
      this.getField('#2').buttonSetIcon(this.getField('\PDFAnim@UseName'+#2cnt).buttonGetIcon());
      } catch(e) {}
      #2PDFAnimUpdateTextProc(#2cnt);
      PDFrun=false;
    }      
    if ((#2running == 0)&&(#2cnt>=(#4-1))&&!(#2eventshift))
    {
      #2cnt=0;
      try {
      this.getField('#2').buttonSetIcon(this.getField('\PDFAnim@UseName'+#2cnt).buttonGetIcon());
      } catch(e) {}
      #2PDFAnimUpdateTextProc(#2cnt);
      PDFrun=false;
    }
    \fi
    if (PDFrun)
    {
      if (#2running == 0)
      { % running==0 not runnning, we start the animation
        #2running = 1;
        try{
          this.getField("#2PlayStop").buttonSetCaption("Pause");
        } catch(e) {}
        if (!(#2eventshift))
        {
          \ifPDFAnim@reverse
          #2lr=false;
          #2PDFAnimKey = app.setInterval('#2PDFAnimate()', #2interval);
          \else
          #2lr=true;
          #2PDFAnimKey = app.setInterval('#2PDFAnimate()', #2interval);
          \fi
        } 
        else 
        {
          \ifPDFAnim@reverse
          #2lr=true;
          #2PDFAnimKey = app.setInterval('#2PDFAnimate()', #2interval);
          \else
          #2lr=false;
          #2PDFAnimKey = app.setInterval('#2PDFAnimate()', #2interval);
          \fi
        }
      }
      else
      { % animation runs, we make pause
        try{
          this.getField("#2PlayStop").buttonSetCaption("Play");
        } catch(e) {}
        #2running = 0; % 0=pause
        app.clearInterval(#2PDFAnimKey);
      }
    }
  };
  function #2PDFAnimUpdateTextProc(framenumber){
    var hodnota= Math.floor(framenumber*100/(\csname LastFrame#2\endcsname-1));
    if (hodnota>100) hodnota=100;
  };
  function #2PDFStopAnimation()
  {
    #2running = 0; % 0=pause
    try{
        app.clearInterval(#2PDFAnimKey);
      } catch(e) {}
    try{
      this.getField("#2PlayStop").buttonSetCaption("Play");
    } catch(e) {}
  };
  function #2PDFAnimate() {
    \ifPDFAnim@step\else % lock fields with percents
    \fi
    if (#2lr ==true) {#2cnt++;} else {#2cnt--;}
    if (#2cnt >= (#4-1)) {% out of upper bound
      \ifPDFAnim@loop
      #2cnt = 0;
      \else
      #2PDFStopAnimation();
      \fi	  
    }
    if (#2cnt < 0) {% out of lower bound
      \ifPDFAnim@loop
      #2cnt = #4 - 1;
      \else
      #2cnt=0;
      #2PDFStopAnimation();
      \fi	  
    }
    #2PDFAnimUpdateTextProc(#2cnt);
    try {
    this.getField('#2').buttonSetIcon(this.getField('\PDFAnim@UseName'+#2cnt).buttonGetIcon());
    } catch (e) {}
  };
}%  
% \fi%
\PDFAnimOpen@def{\thePDFAnimNr}{%
  \ifPDFAnim@remember\else%
  #2cnt = \PDFAnim@defaultframe;
  try{
  this.getField('#2').buttonSetIcon(this.getField('\PDFAnim@UseName'+#2cnt).buttonGetIcon());
  } catch (e) {}
  \fi%
  \ifPDFAnim@step\else%
  \ifPDFAnim@auto 
  if (#2running == 0) {
    #2cnt = \PDFAnim@startframe;
  }
  #2running = 1;
  \ifPDFAnim@reverse
  #2lr=false;
  \else
  #2lr=true;
  \fi
  #2PDFAnimKey = app.setInterval('#2PDFAnimate()', #2interval);
  \fi
  \fi      
}%  
\PDFAnimClose@def{\thePDFAnimNr}{%
  \ifPDFAnim@remember\else%
  if(#2running != 0) 
  {
    #2running = 0;
    app.clearInterval(#2PDFAnimKey);
  }
  #2cnt = \PDFAnim@defaultframe;
  try{
  this.getField('#2').buttonSetIcon(this.getField('\PDFAnim@UseName'+#2cnt).buttonGetIcon());
  } catch (e) {}
  \fi
}%   
\egroup%
\global\addtocounter{PDFAnimNr}{1}%
\xdef\PDFAnimNr{\thePDFAnimNr}%
% 
\ifPDFAnim@debug
\message{PDFAnim: defining animation }\message{\PDFAnimNr}%
\message{with document level javascript:}\message{\csname PDFAnimationJS@\PDFAnimNr@\endcsname}%
\fi%
% 
}



\newcommand\PDFAnimButtons[2][]{%
  \def\PDFAnimPlayButtonInit{\csname #2PDFauto\endcsname}%
  \def\PDFAnimPlayButtonName{#2PlayStop}%
  \@PDFAnimMakeButtons[#1]{#2}%
}


\newcommand{\@PDFAnimMakeButtons}[2][]{%
  \pushButton[#1
  \CA{\string\253}
  \TU{First frame}
  \A{\JS{ 
      #2running=1;
      #2cnt = 0;
      try {
      this.getField('#2').buttonSetIcon(this.getField('#2h'+#2cnt).buttonGetIcon());
      } catch (e) {}
      #2PDFAnimUpdateTextProc(#2cnt);
      #2PDFStopAnimation();
    }}
  ]{#2First}{12pt}{12pt}%
  % 
  \pushButton[#1
  \CA{\string\210}
  \TU{Previous frame}
  \A{\JS{ 
      #2running=1;
      #2cnt --;
      if (#2cnt<0) #2cnt=0;
      try {
      this.getField('#2').buttonSetIcon(this.getField('#2h'+#2cnt).buttonGetIcon());
      } catch (e) {}
      #2PDFAnimUpdateTextProc(#2cnt);
      #2PDFStopAnimation();
    }}
  ]{#2Prec}{12pt}{12pt}%
  %
  \pushButton[#1
  \CA{\PDFAnimPlayButtonInit}
  \TU{Click to Play / Pause animation. Shift+Click plays backwards (if allowed).}
  \A{\JS{#2StartAnimation();}}]{\PDFAnimPlayButtonName}{30pt}{12pt}%
  % 
  \pushButton[#1
  \CA{\string\211}
  \TU{Next frame}
  \A{\JS{
      #2running=1;
      #2cnt ++;
      if (#2cnt>=\csname LastFrame#2\endcsname) #2cnt=(\csname LastFrame#2\endcsname - 1);
      try {
      this.getField('#2').buttonSetIcon(this.getField('#2h'+#2cnt).buttonGetIcon());
      } catch (e) {}
      #2PDFAnimUpdateTextProc(#2cnt);
      #2PDFStopAnimation();
    }}
  ]{#2Next}{12pt}{12pt}%
  % 
  \pushButton[#1
  \CA{\string\273}
  \TU{Last frame}
  \A{\JS{ 
      #2running=1;
      #2cnt = \csname LastFrame#2\endcsname - 1;
      try {
      this.getField('#2').buttonSetIcon(this.getField('#2h'+#2cnt).buttonGetIcon());
      } catch (e) {}
      #2PDFAnimUpdateTextProc(#2cnt);
      #2PDFStopAnimation();
  }}
  ]{#2Last}{12pt}{12pt}%
}


\def\defaultDelayIntName{Delay}

\newcommand\PDFAnimDelayButton[4][]{
  \pushButton
  [#1
  \TU{Click to increase for slower animation, Shift+Click to decrease.}
  \CA{\defaultDelayIntName}
  \RC{\csname #2PDFDelayInterval\endcsname}
  \AA{\AAMouseExit{\JS{
        this.getField("#2Interval").buttonSetCaption("\defaultDelayIntName");
      }}}
  \A{\JS{      
      if (event.shift) #2interval=#2interval-5;
      else #2interval=#2interval+5;
      if (#2interval<=0) #2interval = 5;
      this.getField("#2Interval").buttonSetCaption(#2interval,2);
      this.getField("#2Interval").buttonSetCaption(#2interval);
      if (#2running==1)
      {
        app.clearInterval(#2PDFAnimKey);
        #2PDFAnimKey = app.setInterval('#2PDFAnimate(true)', #2interval);
      }
    }}
  ]
  {#2Interval}{#3}{#4}}


\makeatother


%%% Tady konci upravy vnitrnich maker pro pdfanim. Snad budou vsechny
%%% option fungovat.
%%% ------------Cut here------------


